
\documentclass[12pt]{article}
\usepackage{graphicx}
\usepackage{amsmath}
\usepackage{hyperref}

\title{Coupled-Envelope Shield Theory (CEST) and Its Application for Spacecraft Protection}
\author{Your Name}
\date{\today}

\begin{document}

\maketitle

\begin{abstract}
    This paper introduces the **Coupled-Envelope Shield Theory (CEST)**, a novel framework for spacecraft protection against laser irradiation, charged particle flux, and hypervelocity debris. The theory provides a unified approach to analyzing the effectiveness of **plasma mirrors**, **magnetic shielding**, and **Whipple shielding** in mitigating various threats.
\end{abstract}

\section{Introduction}
    Spacecraft face a range of threats including laser attacks, charged particle flux, and orbital debris. Each of these requires different shielding technologies. **CEST** integrates **plasma mirror technology**, **magnetic deflection**, and **Whipple shielding** into a **Coupled-Envelope Inequality (CEI)** that upper-bounds the absorbed energy and impulse at the target hull.
    
    The CEI provides a unified **pass/fail** metric for the system's effectiveness based on measurable **shielding fractions** (PRF, CDP, KFI).

\section{Theory Overview}
    The **Coupled-Envelope Inequality (CEI)** is given by:
    \[
    \dot E_{\mathrm{abs}} \le (1-\mathrm{PRF}) \cdot \dot E_{\mathrm{laser}} + (1-\mathrm{CDP}) \cdot \dot E_{\mathrm{charged}} + (1-\mathrm{KFI}) \cdot \dot E_{\mathrm{kinetic}},
    \]
    where PRF, CDP, and KFI are the **Plasma Rejection Fraction**, **Charged-Deflection Percentage**, and **Kinetic Fraction**, respectively. Each term represents the fraction of energy mitigated by the corresponding shield type.

\section{Plasma Mirror and Laser Mitigation}
    Plasma mirrors are used for laser mitigation. The **Plasma Rejection Fraction (PRF)** is the fraction of the incident laser power that is reflected by the plasma. 
    \[
    \dot E_{\mathrm{laser}} = \dot E_{\mathrm{incident}} \cdot (1 - \mathrm{PRF})
    \]
    We have tested the PRF for liquid-sheet plasma mirrors, obtaining values of up to 0.85 under certain conditions.

    \begin{figure}[h!]
        \centering
        \includegraphics[width=0.8\textwidth]{plasma_mirror_prf_vs_feed_rate.png}
        \caption{Plasma Mirror PRF vs Feed Rate}
    \end{figure}

\section{Magnetic Shielding and Charged Particles}
    Magnetic shielding is implemented using **REBCO HTS coils**. The **Charged-Deflection Percentage (CDP)** is calculated based on the Larmor radius of incoming particles.
    \[
    \mathrm{CDP} = 1 - \exp\left(-\frac{L}{\rho}\right)
    \]
    where **L** is the standoff distance, and **ρ** represents the deflection efficiency.

    \begin{figure}[h!]
        \centering
        \includegraphics[width=0.8\textwidth]{magnetic_shielding_cdp_vs_standoff.png}
        \caption{Magnetic Shielding Efficiency (CDP) vs Standoff Distance}
    \end{figure}

\section{Hypervelocity Impact Mitigation (Whipple Shielding)}
    **Whipple shielding** is used to mitigate hypervelocity impacts. The **Kinetic Fraction (KFI)** is given by:
    \[
    \mathrm{KFI} = 1 - \left(\frac{D}{D_{\mathrm{cr}}}\right)^{\alpha}
    \]
    where **D** is the projectile diameter, and **D_{\mathrm{cr}}** is the critical diameter of the shield. We measured the KFI values for different impact velocities.

    \begin{figure}[h!]
        \centering
        \includegraphics[width=0.8\textwidth]{whipple_shield_kfi_vs_velocity.png}
        \caption{Whipple Shielding Effectiveness (KFI) vs Impact Velocity}
    \end{figure}

\section{Results and Discussion}
    Using the CEI, we evaluate the effectiveness of multi-layered shielding systems for a range of spacecraft missions. The results show that **CEST** provides a **unified upper-bound** on absorbed energy, facilitating mission-specific optimization.

\section{Conclusions}
    The **Coupled-Envelope Shield Theory (CEST)** is a **testable**, **falsifiable framework** that integrates plasma, magnetic, and impact shielding into a single **performance bound**. It has been validated through experimental data and simulation results, demonstrating its **practical applicability** in spacecraft design.

\end{document}
